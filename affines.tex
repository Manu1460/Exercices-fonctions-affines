\documentclass[12pt,a4paper]{article}
\usepackage[utf8]{inputenc}
\usepackage[T1]{fontenc}
\usepackage{amsmath}
\usepackage{amsfonts}
\usepackage{amssymb}
\usepackage{multicol}
\usepackage{qrcode}
\usepackage{lmodern}
\usepackage{colortbl}%permet de griser les cases
\usepackage{tabularx, multirow}
%\usepackage{lscape}
\usepackage{xcolor}
%\usepackage{graphicx}
\usepackage{tikz,tkz-base}
% Fichier de style stage.sty [UTF8]
% Copyleft Laurent Bretonnière, laurent.bretonniere@gmail.com
% Version du 14/03/2015 

%*********************************************************************************************
% Packages
%*********************************************************************************************

\usepackage[utf8]{inputenc}%			encodage du fichier source (Linux)
\usepackage[TS1,T1]{fontenc}%			gestion des accents (pour les pdf)
\usepackage[french]{babel}%				rajouter éventuellement greek, etc.
\frenchbsetup{CompactItemize=false,StandardLists=true}
\usepackage{enumitem}%
\setenumerate[1]{label=\arabic*/}%
\setenumerate[2]{label=\alph*/}%
%\setlist{font=\bfseries,leftmargin=*}%
\setlist{font=\bfseries,leftmargin=*,topsep=1pt,partopsep=1pt,itemsep=2pt,parsep=1pt}%

\usepackage{textcomp}%					caractères additionnels
\usepackage{amsmath,amssymb}%			pour les maths (1)
\usepackage{amsfonts}%					pour les maths (2)
\usepackage{lmodern}%					remplacer éventuellement par txfonts, fourier, etc.

\usepackage{graphicx}%

% cf site web http://www.khirevich.com/latex/microtype/
%\usepackage[babel=true,kerning=true]{microtype}%
\usepackage{microtype}%

\usepackage{dsfont}%					pour les ensembles de nombres N,Z,D,Q,R,C ...
\usepackage{mathrsfs}%					pour les écritures calligraphiques (genre Cf et Df)
\usepackage[np]{numprint}% page 58  	pour afficher les nombres 3 par 3
\usepackage[e]{esvect}% page 60		vecteurs
\usepackage{stmaryrd}%					pour les intervalles entiers et sslash
\usepackage{empheq}% 					encadrer en mode math
\usepackage{xcolor}%					pour gérer les couleurs
\usepackage{soul}% 						pour les fluos	
\usepackage{xspace}%					gestion des espaces
\usepackage{pifont}% 					trèfle, pique, carreau, coeur
\usepackage{eurosym}%					symbole euro
\usepackage{mathabx}%					choix personnel

\usepackage{hyperref}%
\hypersetup{%
colorlinks=true,%
breaklinks=true,%
citecolor=red,%
urlcolor=blue,%
linkcolor=black,% ou blue
bookmarksopen=false,%
pdfcreator=PDFLaTeX,%
pdfproducer=PDFLaTeX,%
pdfmenubar=true,%
pdftoolbar=true,%
pdfauthor={Laurent Bretonnière},%
pdfkeywords={Mathématiques},%
pdfstartview=XYZ}%

%*********************************************************************************************
% Mathématiques (cf chapitre 7 page 56, LaTeX pour le prof de maths, IREM de Lyon)
%*********************************************************************************************

% Fonctions usuelles
\DeclareMathOperator{\cotan}{cotan}%
\DeclareMathOperator{\ch}{ch}%
\DeclareMathOperator{\sh}{sh}%
\DeclareMathOperator{\thyp}{th}%
\renewcommand{\th}{\thyp}%
\DeclareMathOperator{\Arcsin}{Arcsin}%
\DeclareMathOperator{\Arccos}{Arccos}%
\DeclareMathOperator{\Arctan}{Arctan}%
\DeclareMathOperator{\Argsh}{Argsh}%
\DeclareMathOperator{\Argch}{Argch}%
\DeclareMathOperator{\Argth}{Argth}%
\DeclareMathOperator{\pgcd}{pgcd}%
\DeclareMathOperator{\ppcm}{ppcm}%
\DeclareMathOperator{\card}{card}%

% Composée de fonctions
\newcommand{\rond}{\circ}%

% Multiplication
\newcommand{\x}{\times}

% Constantes usuelles
\renewcommand{\i}{\mathrm{i}}%
\newcommand{\e}{\mathrm{e}}%

% Éléments différentiels
\newcommand{\dt}{\,\textrm{d}t}%
\newcommand{\du}{\,\textrm{d}u}%
\newcommand{\dv}{\,\textrm{d}v}%
\newcommand{\dw}{\,\textrm{d}w}%
\newcommand{\dx}{\,\textrm{d}x}%
\newcommand{\dy}{\,\textrm{d}y}%
\newcommand{\dz}{\,\textrm{d}z}%

% Ensembles de nombres
\newcommand{\ensnb}[1]{\ensuremath{\mathbb{#1}}}%
\newcommand{\N}{\ensnb{N}}%
\newcommand{\Z}{\ensnb{Z}}%
\newcommand{\D}{\ensnb{D}}%
\newcommand{\Q}{\ensnb{Q}}%
\newcommand{\R}{\ensnb{R}}%
\newcommand{\Rp}{\R_{+}}%
\newcommand{\Rm}{\R_{-}}%
\newcommand{\mtsmall}{\fontsize{5pt}{5pt}\selectfont}%
\newcommand{\Rpe}{\R_{\mbox{\mtsmall$+$}}^{\mskip0.4mu\ast}}%
\newcommand{\Rme}{\R_{\mbox{\mtsmall$-$}}^{\mskip0.4mu\ast}}%
\newcommand{\Ret}{\R^{\ast}}% \Re pris pour la partie réelle d'un complexe
\newcommand{\Ne}{\N^{\ast}}%
\newcommand{\Ze}{\Z^{\ast}}%
\newcommand{\C}{\ensnb{C}}%
\newcommand{\Ce}{\C^{\ast}}%

% Noms de points en majuscules en romain (et non pas en italiques)
\DeclareMathSymbol{A}{\mathalpha}{operators}{`A}
\DeclareMathSymbol{B}{\mathalpha}{operators}{`B}
\DeclareMathSymbol{C}{\mathalpha}{operators}{`C}
\DeclareMathSymbol{D}{\mathalpha}{operators}{`D}
\DeclareMathSymbol{E}{\mathalpha}{operators}{`E}
\DeclareMathSymbol{F}{\mathalpha}{operators}{`F}
\DeclareMathSymbol{G}{\mathalpha}{operators}{`G}
\DeclareMathSymbol{H}{\mathalpha}{operators}{`H}
\DeclareMathSymbol{I}{\mathalpha}{operators}{`I}
\DeclareMathSymbol{J}{\mathalpha}{operators}{`J}
\DeclareMathSymbol{K}{\mathalpha}{operators}{`K}
\DeclareMathSymbol{L}{\mathalpha}{operators}{`L}
\DeclareMathSymbol{M}{\mathalpha}{operators}{`M}
\DeclareMathSymbol{N}{\mathalpha}{operators}{`N}
\DeclareMathSymbol{O}{\mathalpha}{operators}{`O}
\DeclareMathSymbol{P}{\mathalpha}{operators}{`P}
\DeclareMathSymbol{Q}{\mathalpha}{operators}{`Q}
\DeclareMathSymbol{R}{\mathalpha}{operators}{`R}
\DeclareMathSymbol{S}{\mathalpha}{operators}{`S}
\DeclareMathSymbol{T}{\mathalpha}{operators}{`T}
\DeclareMathSymbol{U}{\mathalpha}{operators}{`U}
\DeclareMathSymbol{V}{\mathalpha}{operators}{`V}
\DeclareMathSymbol{W}{\mathalpha}{operators}{`W}
\DeclareMathSymbol{X}{\mathalpha}{operators}{`X}
\DeclareMathSymbol{Y}{\mathalpha}{operators}{`Y}
\DeclareMathSymbol{Z}{\mathalpha}{operators}{`Z}

% Raccourci displaystyle + hack :-)
\newcommand{\dps}{\displaystyle}%
\newcommand{\dsp}{\displaystyle}%
\newcommand{\disp}{\displaystyle}%
\everymath{\displaystyle}%

% Mots usuels en mode math
\newcommand{\mtext}[1]{\quad\text{#1}\quad}%
\newcommand{\et}{\mtext{et}}%
\newcommand{\ou}{\mtext{ou}}%
\newcommand{\si}{\mtext{si}}%

% Flèches
\newcommand{\tv}{\shortrightarrow}% tend vers
\renewcommand{\to}{\shortrightarrow}% tend vers
\newcommand{\suit}{\hookrightarrow}% X suit la loi...
\newcommand{\dans}{\longrightarrow}% f:\R\dans\R
\newcommand{\donne}{\longmapsto}% f:x\donne 2x+3
\newcommand{\ppv}{\leftarrow}% flèche <-- d'affectation "prend pour valeur"
\newcommand{\ech}{\leftrightarrow}% double flèche <--> : échange/swap

% Vecteurs
% \vv{AB} en utilisant l'extension \usepackage[e]{esvect}

% Norme et valeur absolue
\newcommand{\abs}[1]{\left\lvert#1\right\rvert}%
\newcommand{\norme}[1]{\left\lVert#1\right\rVert}%

% Complexes
\renewcommand{\Re}{\operatorname{Re}}
\renewcommand{\Im}{\operatorname{Im}}
\renewcommand{\bar}{\overline}

% Matrices
\newcommand{\trans}[1]{{\vphantom{#1}}^{\mathit{t}}\!{#1}}%

% Coefficient binomial
\newcommand{\cb}[2]{\binom{#2}{#1}}%

% Matrice augmentée
\makeatletter
\renewcommand*\env@matrix[1][*\c@MaxMatrixCols c]{%
  \hskip -\arraycolsep
  \let\@ifnextchar\new@ifnextchar
  \array{#1}}
\makeatother

% Parallèles et perpendiculaires
\newcommand{\para}{\sslash}%
% perp pour perpendiculaire

% Intervalles
\newcommand{\intervalle}[4]{\mathchoice%
{\left#1#2\mathclose{}\mathpunct{},#3\right#4}% mode \displaystyle
{\mathopen{#1}#2\mathclose{}\mathpunct{},#3\mathclose{#4}}% mode \textstyle
{\mathopen{#1}#2\mathclose{}\mathpunct{},#3\mathclose{#4}}% mode \scriptstyle
{\mathopen{#1}#2\mathclose{}\mathpunct{},#3\mathclose{#4}}% mode \scriptscriptstyle
}%

\newcommand{\intff}[2]{\intervalle{[}{#1}{#2}{]}}%
\newcommand{\intof}[2]{\intervalle{]}{#1}{#2}{]}}%
\newcommand{\intfo}[2]{\intervalle{[}{#1}{#2}{[}}%
\newcommand{\intoo}[2]{\intervalle{]}{#1}{#2}{[}}%

% Ancienne configuration
%\newcommand{\intervalle}[4]{\mathopen{#1}#2\mathclose{}\mathpunct{},#3\mathclose{#4}}%
%\newcommand{\intoo}[2]{\ensuremath{\,\left]  #1 \,, #2  \right[\, }}%
%\newcommand{\intof}[2]{\ensuremath{\,\left]  #1 \,, #2  \right]\, }}%
%\newcommand{\intfo}[2]{\ensuremath{\,\left[  #1 \,, #2  \right[\, }}%
%\newcommand{\intff}[2]{\ensuremath{\,\left[  #1 \,, #2  \right]\, }}%

% Intervalles entiers
\newcommand{\intn}[2]{\intervalle{\llbracket}{#1}{#2}{\rrbracket}}%

% Ensembles et Probabilités
\newcommand{\vide}{\varnothing}% ensemble vide
\newcommand{\union}{\cup}%
\newcommand{\inter}{\cap}%
\newcommand{\Union}{\bigcup}%
\newcommand{\Inter}{\bigcap}%
\newcommand{\compl}{\complement}% complémentaire
\newcommand{\inclus}{\subseteq}% inclus : je n'aime pas \subset je préfère \subseteq...
\newcommand{\inclusstrict}{\subsetneq}% inclus au sens strict ...
\newcommand{\contient}{\supseteq}% contient
\newcommand{\contientstrict}{\supsetneq}% contient au sens strict ...
\newcommand{\prive}{\setminus}% privé de ...
\renewcommand{\P}{\mathrm{P}} % probabilité
\newcommand{\V}{\mathrm{V}} % variance
\newcommand{\E}{\mathrm{E}} % espérance

% ensemble des ... tels que ...
\newcommand{\enstq}[2]{\left\{#1\,\;\middle|\;\,#2\right\}}%

% Pointillés anti-diagonale
\newcommand{\adots}{\mathinner{\mkern2mu\raise 1pt\hbox{.}\mkern3mu\raise 4pt\hbox{.}\mkern1mu\raise 7pt\hbox{.}}}%

% Partie entière
\newcommand{\ent}[1]{\left\lfloor#1\right\rfloor}% partie entière (première notation)
\newcommand{\Ent}[1]{\textrm{Ent}\mathopen{}\left(#1\right)}% partie entière (deuxième notation)

% Angle
\renewcommand{\angle}{\widehat}%

% Limites
\newcommand{\iy}{\infty}% 
\newcommand{\ii}{\infty}%
\newcommand{\zp}{0^{+}}%
\newcommand{\zm}{0^{-}}%

% Encadrement d'une formule
%\begin{empheq}[box=\fbox]{equation*}
% ...    
%\end{empheq}

% Couleurs
% http://www.latextemplates.com/svgnames-colors
\definecolor{bleu1}{HTML}{000080}%
\definecolor{grispale}{RGB}{245 245 245}%
\definecolor{bistre}{rgb}{.75 .50 .30}%
\definecolor{grisclair}{gray}{0.8}%
\definecolor{bleuclair}{rgb}{0.7, 0.7, 1.0}%
\definecolor{rosepale}{rgb}{1.0, 0.7, 1.0}%

% Fluos !
\newcommand{\fluo}[1]{\sethlcolor{rosepale}\hl{#1}}%

% Lettres calligraphiées
\newcommand{\Cf}{\mathscr{C}_f}%
\newcommand{\Df}{\mathscr{D}_f}%
\newcommand{\Cg}{\mathscr{C}_g}%
\newcommand{\Dg}{\mathscr{D}_g}%
\newcommand{\Ch}{\mathscr{C}_h}%
\newcommand{\Dh}{\mathscr{D}_h}%

% Degré
\newcommand{\Degre}{\ensuremath{^\circ}}

% Lettres grecques
\renewcommand{\epsilon}{\varepsilon}%
\renewcommand{\phi}{\varphi}%

% Mots usuels
\newcommand{\ie}{\;\textit{i.e.}\;\xspace}
\newcommand{\cad}{c'est--à--dire\xspace}%
\newcommand{\pourcent}{\unskip~\%\xspace}%
\newcommand{\ssi}{si et seulement si\xspace}%
\newcommand{\eve}{événement\xspace}%
\newcommand{\eves}{événements\xspace}%
\newcommand{\sev}{sous-espace vectoriel\xspace}%
\newcommand{\ipp}{intégration par parties\xspace}%
\newcommand{\iaf}{inégalité des accroissements finis\xspace}%
\newcommand{\tvi}{théorème des valeurs intermédiaires\xspace}%
\newcommand{\fpt}{formule des probabilités totales\xspace}%
\newcommand{\fpc}{formule des probabilités composées\xspace}%
\newcommand{\sce}{système complet d'événements\xspace}%
\newcommand{\srld}{suite récurrence linéaire d'ordre $2$\xspace}%
\newcommand{\sag}{suite arithmético-géométrique\xspace}%

% Guillements français
\newcommand{\guill}[1]{%
\og{}#1\fg{}}%


% Inégalités
\renewcommand{\leq}{\leqslant}%
\renewcommand{\geq}{\geqslant}%
\renewcommand{\le}{\leqslant}%
\renewcommand{\ge}{\geqslant}%
\newcommand{\pg}{\geqslant}%
\newcommand{\pp}{\leqslant}%

% Environ
\newcommand{\environ}{\simeq}%
\renewcommand{\approx}{\simeq}%


% Tableau de variations en TikZ
\usepackage{tikz,tkz-tab}%
\definecolor{fondpaille}{rgb}{1,1,1}%

% Aire d'une figure géométrique
\newcommand{\aire}{\text{aire}}%

% Paramétrage de quelques variables
\setlength{\columnsep}{1cm}%
\setlength{\columnseprule}{0.4pt}%
\setlength{\parindent}{0pt}%

% Mathématiciens
\newcommand{\GJ}{Gauss\,--\,Jordan\xspace}%
\newcommand{\KH}{König\,--\,Huygens\xspace}%

% Siècle en lettres romains
\newcommand{\siecle}[1]{\textsc{\romannumeral #1}\textsuperscript{e}~si\`ecle}%

% Couleurs jeu de carte
\newcommand{\pique}{\ding{171}}%
\newcommand{\coeur}{\ding{170}}%
\newcommand{\carreau}{\ding{169}}%
\newcommand{\trefle}{\ding{168}}%

% Exercices (fiche)
\renewcommand*{\hrulefill}[2][0pt]{\leavevmode \leaders \hbox to 1pt{\rule[#1]{1pt}{#2}} \hfill \kern 0pt}%

\newcounter{numexercice}%

\newenvironment{exercice}{\stepcounter{numexercice}\ovalbox{\textbf{\thenumexercice}}\hrulefill[3pt]{0.5pt}\par\medskip\nopagebreak[4]}{\medskip}

% Trait de la largeur de la feuille
\newcommand{\trait}{\hbox{\raisebox{0.4em}{\vrule depth 0pt height 0.4pt width \textwidth}\linebreak}}%

\newcommand{\demitrait}{\hbox{\raisebox{0.4em}{\vrule depth 0pt height 0.4pt width 0.48\textwidth}\linebreak}}%

\newcommand{\LV}{Lycée Le Verrier, Saint\,--\,Lô}%
%\newcommand{\itb}{\item[\textbullet]}%
\newcommand{\itb}{\item}%
%\newcommand{\Gaffe}{\ding{54}\ding{54}\ding{54}\quad}%
\newcommand{\gaffe}{\ding{56}\ding{56}\ding{56}\quad}%


% Fichier de style stage2.sty [UTF8]
% Copyleft Laurent Bretonnière, laurent.bretonniere@gmail.com
% Version du 16/03/2015

\usepackage{mathtools}%	
\usepackage{fancybox}%
\usepackage{lastpage}%

\usepackage{fancyhdr}%
\renewcommand{\headrulewidth}{0.8pt}%
\renewcommand{\footrulewidth}{0.8pt}%

\usepackage[tikz]{bclogo}%
\renewcommand\bcStyleTitre[1]{\normalsize\textbf{#1}\smallskip}%
\renewcommand\logowidth{0pt}%

\newcommand{\fin}{\begin{center}%
$\clubsuit\clubsuit\clubsuit$%
\end{center}}%

\newcommand{\un}{\ding{192}\xspace}%
\newcommand{\deux}{\ding{193}\xspace}%
\newcommand{\trois}{\ding{194}\xspace}%
\newcommand{\quatre}{\ding{195}\xspace}%
\newcommand{\cinq}{\ding{196}\xspace}%
\newcommand{\six}{\ding{197}\xspace}%
\newcommand{\sept}{\ding{198}\xspace}%
\newcommand{\huit}{\ding{199}\xspace}%
\newcommand{\neuf}{\ding{200}\xspace}%

\setlength{\headheight}{15pt}%

%*********************************************************************************************
% Cours
%*********************************************************************************************

\usepackage[Lenny]{fncychap}%
\ChNumVar{\fontsize{76}{80}\usefont{OT1}{pzc}{m}{n}\selectfont}%
\ChTitleVar{\raggedleft\Huge\sffamily\bfseries}%

\renewcommand{\thesection}{\Roman{section})}%
\renewcommand{\thesubsection}{\arabic{subsection})}%
\renewcommand{\thesubsubsection}{\alph{subsubsection})}%

%*********************************************************************************************
% Environnements prédéfinis BCLOGO
%*********************************************************************************************

%% Lemme
\newenvironment{lem}{\begin{bclogo}[couleurBord=black!50,arrondi=0.1,logo=\hspace{17pt},barre=none]{Lemme :}}{\end{bclogo}\medskip}%

%% Proposition
\newenvironment{prop}[1][]{\begin{bclogo}[couleurBord=black!50,arrondi=0.1,logo=\hspace{17pt},barre=none]{Proposition :~#1}}{\end{bclogo}\medskip}%

%% Théorème
\newlength{\textlarg}
\settowidth{\textlarg}{~}
\newenvironment{theo}[1][\hspace{-\textlarg} :]{\begin{bclogo}[couleur=black!5,couleurBord=black!50,arrondi=0.1,logo=\hspace{17pt}, barre=none]{Théorème~#1}}{\end{bclogo}\medskip}%

\newenvironment{theon}[1][]{\begin{bclogo}[couleur=black!5,couleurBord=black!50,arrondi=0.1,logo=\hspace{17pt}, barre=none]{Théorème :~#1}}{\end{bclogo}\medskip}%

%% Corollaire
\newenvironment{coro}[1][]{\begin{bclogo}[couleurBord=black!50,arrondi=0.1,logo=\hspace{17pt},barre=none]{Corollaire :~#1}}{\end{bclogo}\medskip}%

%% Définition(s)

\newenvironment{defi}{\begin{bclogo}[couleurBord=black!50,arrondi=0.1,logo=\hspace{17pt}, barre=none]{Définition :}}{\end{bclogo}\medskip}%

\newenvironment{defis}{\begin{bclogo}[couleurBord=black!50,arrondi=0.1,logo=\hspace{17pt}, barre=none]{Définitions :}}{\end{bclogo}\medskip}%

%% Preuve
\newenvironment{pf}{\renewcommand\logowidth{17pt}\begin{bclogo}[noborder=true,logo=\hspace{17pt},couleurBarre=black!25,epBarre=3.5]{Preuve :}}{\hspace*{\fill}$\Box$\end{bclogo}\smallskip\renewcommand\logowidth{0pt}}%

%\blacksquare

%% Notation
\newenvironment{nota}{\begin{bclogo}[couleurBord=black!50,arrondi=0.1,logo=\hspace{17pt},barre=none]{Notation :}}{\medskip}%

%% Exercice et Exercice-type
\newenvironment{exo}{$\circledast$ \quad\textsc{\underline{exercice} :}~}{\hspace*{\fill}$\circledast$\vskip 8pt}
\newenvironment{type}{$\blacktriangleright$ \quad\textsc{exercice-type :}~}{\hspace*{\fill}$\blacktriangleleft$\vskip 8pt}

%% Exemple(s)
\newenvironment{exem}{\textbf{Exemple :}~}{\medskip}
\newenvironment{exems}{\textbf{Exemples :}~}{\medskip}

%% Remarque(s)
\newenvironment{rem}{\textbf{Remarque :}~}{\medskip}
\newenvironment{rems}{\textbf{Remarques :}~}{\medskip}

%% Rappel(s)
\newenvironment{rap}{\textbf{Rappel :}~}{\medskip}
\newenvironment{raps}{\textbf{Rappels :}~}{\medskip}

%% Cas particulier(s)
\newenvironment{cp}{\textbf{Cas particulier :}~}{\medskip}
\newenvironment{cps}{\textbf{Cas particuliers :}~}{\medskip}

%% Application
\newenvironment{appli}{\textbf{Application :}~}{\medskip}%{\medskip} 

%******************************************

%Permet le code python sur lateX
\usepackage{minted}
\usemintedstyle{lovelace}

%box exercice
\usepackage{tcolorbox}
\newtcolorbox{mybox}[1]{colback=yellow!5!,colframe=yellow!50!black,colbacktitle=yellow!75!black,fonttitle=\bfseries,
title=#1}

%%Propriété
\newenvironment{pro}[1][]{\begin{bclogo}[couleurBord=black!50,arrondi=0.1,logo=\hspace{17pt},barre=none]{Propriété :~#1}}{\end{bclogo}\medskip}%




\usepackage[left=2cm,right=2cm,top=2cm,bottom=2cm]{geometry}
\def\Oij{$\left(\text{O},~\vec{i},~\vec{j}\right)$}
\usepackage{fancyhdr}
\usepackage{MnSymbol,wasysym}

%Permet le code python sur lateX
\usepackage{minted}
\usemintedstyle{lovelace}



\begin{document}
\textbf{2nd} \hfill \textbf{Fonctions affines} \hfill Lycée Jean Rostand\\
\trait 

\subsection*{Exercice 1}

Parmi les fonctions suivantes définies sur $\R$, dire lesquelles sont affines. Justifier la réponse.

\begin{enumerate}
\begin{minipage}[t]{0.4\linewidth}
\item $f(x)=\sqrt{2}x+1$
\item $g(x)=\dfrac{2}{3x}+1$
\end{minipage}
\begin{minipage}[t]{0.4\linewidth}
\item $h(x)=5(x+1)$
\item $i(x)=-3x$
\end{minipage}
\begin{minipage}[t]{0.4\linewidth}
\item $j(x)=2x^2+3$
\item $k(x)=(x+1)^2-x^2$
\end{minipage}
\begin{minipage}[t]{0.4\linewidth}
\item $l(x)=4$

\end{minipage}
\end{enumerate}

\subsection*{Exercice 2}
Les expressions suivantes définissent-elles une fonction affine ? (justifier et, si oui, préciser le coefficient directeur) :\medskip\\
$f(x)=-5x^2+7x-1$\hfill $g(x)=\frac{3x-5}{7}$\hfill $h(x)=(x-3)(2x-5)-2x^2$
\subsection*{Exercice 3}

Dans un même repère \Oij{}, tracer la courbe représentative de chacune des fonctions suivantes définies sur $\R$.


\begin{enumerate}
\begin{minipage}[c]{0.3\linewidth}
\item $f(x)=2x-3$
\end{minipage}
\begin{minipage}[c]{0.3\linewidth}
\item $g(x)=-x+4$
\end{minipage}
\begin{minipage}[c]{0.3\linewidth}
\item $h(x)=\dfrac{1}{2}x-2$
\end{minipage}
\begin{minipage}[c]{0.3\linewidth}
\item $i(x)=-3x$

\end{minipage}
\end{enumerate}


\subsection*{Exercice 4}

\begin{enumerate}
    \item Dans un même repère, représenter les fonctions $f$ et $g$ définie sur $\R$ par $f(x)=-2x+5$ et $g(x)=\dfrac{1}{2}x$
    \item Résoudre graphiquement puis par le calcul l'équation $f(x)=g(x)$
\end{enumerate}

\subsection*{Exercice 5}

\begin{enumerate}
    \item Déterminer graphiquement l'expression de la fonction de affine $f_1$ dont la représentation graphique est la droite $\mathscr{D}{_1}$
    \item Déterminer graphiquement l'expression de la fonction de affine $f_2$ dont la représentation graphique est la droite $\mathscr{D}{_2}$
    \item Tracer la droite qui représente la fonction affine $f_3$ définie par $f_3(x)=-\dfrac{5}{4}x+4$.
    \item Déterminer par le calcul l'expression de la fonction affine $f_4$ dont la représentation graphique est la droite $(EF)$ avec $E(-1;-3)$ et $F(1;1)$.
\end{enumerate}



\begin{center}
\begin{tikzpicture}[scale=0.8]

\draw [gray,xstep=1,ystep=1] (-7,-6) grid (10,7);
\draw [->] (-7,0) -- (10,0) node [below left] {$x$};
\draw [->] (0,-6) -- (0,7) node [below left] {$y$};

\foreach \x in {-7,...,9}
\draw (\x,0.5mm) -- (\x,-0.5mm) node [below] {$\x$};
\foreach \y in {-6,...,6}
\draw (0.5mm,\y) -- (-0.5mm,\y) node [left] {$\y$};


\draw [domain=-7:7.5,line width=1pt,samples=100] plot (\x,{2/3*\x+2});
\draw [domain=-4:2.5,line width=1pt,samples=100] plot (\x,{-2*\x-1});
\draw (7,5.7) node [right] {$\mathscr{D}_{1}$};
\draw (2,-3) node [right] {$\mathscr{D}_{2}$};

\draw (-1,-3) node{$\bullet$};
\draw (-1,-3) node[left]{$E$};
\draw (1,1) node{$\bullet$};
\draw (1,1) node[right]{$F$};

\end{tikzpicture}
\end{center}











\subsection*{Exercice 6}
\begin{enumerate}
\item Construire, dans le repère de la question /2, les représentations graphiques des fonctions définies par $f(x)=3x-2$, $g(x)=-\frac{3}{4}x+5$, $h(x)=-3$ et $k(x)=2x$.
\item Donner l'expression de chaque fonction associée aux droites suivantes:\\
\end{enumerate}
\begin{center}
\begin{tikzpicture}[scale=0.8]
\draw [gray,xstep=1,ystep=1] (-4,-4) grid (10,6);
\draw [->,>=latex] (-4,0) -- (10,0) node [below left] {$x$};
\draw [->,>=latex] (0,-4) -- (0,6) node [below left] {$y$};

\foreach \x in {-4,...,9}
\draw (\x,0.5mm) -- (\x,-0.5mm) node [below] {$\x$};
\foreach \y in {-4,...,5}
\draw (0.5mm,\y) -- (-0.5mm,\y) node [left] {$\y$};

%\draw (0,0) node [below left] {$O$};

\clip (-4,-4) rectangle (10,6);
\draw [domain=-4:10,samples=100] plot (\x,{-(1/3)*\x+4});
\draw [domain=-4:10,samples=100] plot (\x,{-1*\x});
\draw [domain=-4:10,samples=100] plot (\x,{(3/4)*\x-2});
\draw [domain=-4:10,samples=100] plot (\x,{4});
\draw (-1.5,4.7) node [right] {$\mathscr{D}_{1}$};
\draw (8.1,4.8) node [right] {$\mathscr{D}_{2}$};
\draw (4.3,4.2) node [right] {$\mathscr{D}_{3}$};
\draw (3.3,-3.2) node [right] {$\mathscr{D}_{4}$};
\end{tikzpicture}
\end{center}

\subsection*{Exercice 7}
Déterminer l'expression de la fonction affine $f$ telle que $f(-2)=4$ et $f(5)=7$.

\subsection*{Exercice 8}

On sait que $f$ est une fonction affine.\\
Dans chacun des cas et lorsque cela est possible, exprimer $f(x)$ en fonction de $x$

\begin{enumerate}
\begin{minipage}[c]{0.4\linewidth}
\item $f(-2)=-1$ et $f(4)=2$
\item $f(-1)=5$ et $f(-2)=1$
\end{minipage}
\begin{minipage}[c]{0.4\linewidth}
\item $f(6)=4$ et $f(-9)=-6$
\item $f(-1)=2$ et $f(3)=2$
\end{minipage}
\begin{minipage}[c]{0.4\linewidth}
\item $f(1)=\dfrac{1}{2}$ et $f\left(\dfrac{3}{2}\right)=-\dfrac{1}{2}$
\item $f(1)=-1$ et $f(1)=2$
\end{minipage}
\end{enumerate}

\subsection*{Exercice 9}

D'après ce tableau de valeurs, la fonction $g$, définie sur $\R$, peut-elle être une fonction affine ?

\begin{center}
 { \setlength{\tabcolsep}{9mm}
\begin{tabular}{|c|c|c|c|c|} \hline
$x$ & $1$&$3$& $3,5$&$4,5$ \\ \hline
$g(x)$ &$-5$ & $-1$&$0$&$3$  \\ \hline
\end{tabular} }
   
\end{center}

\subsection*{Exercice 10}
Dans chacun des cas suivants établir le tableau de signes de la fonction $f$.

\begin{enumerate}
\begin{minipage}[c]{0.4\linewidth}
\item $f(x)=2x-4$
\item $f(x)=-3x+2$
\end{minipage}
\begin{minipage}[c]{0.4\linewidth}
\item $f(x)=4x$
\item $f(x)=-5x+7$
\end{minipage}
\begin{minipage}[c]{0.4\linewidth}
\item $f(x)=3x+4$ 
\item $f(x)=-7$
\end{minipage}
\end{enumerate}


\subsection*{Exercice 11}
Soient $f$ et $g$ les fonctions définies par $f(x)=5x-2$ et $g(x)=\frac{5}{3}x+6$.\medskip
\begin{enumerate}
\item Résoudre l'équation $f(x)=g(x)$. En déduire le point d'intersection entre les courbes $C_{f}$ et $C_{g}$.\medskip
\item Le point de coordonnées $(2;7)$ se trouve-t-il sur la courbe de $f$ ?\medskip 
\end{enumerate}

\subsection*{Exercice 12}

Dresser le tableau de signes des expressions suivantes:
\begin{enumerate}
    \item $A(x)=x+7$
    \item $B(x)=5-x$
\end{enumerate}


\subsection*{Exercice 13}

Dresser le tableau de signes des expressions suivantes:
\begin{enumerate}
    \item $A(x)=3x+4$
    \item $B(x)=-5x+8$
\end{enumerate}

\subsection*{Exercice 14}

\begin{enumerate}
    \item Etudier le signe des expressions $A(x)=4x-12$ et $B(x)=-3x+7$
    \item En déduire le signe du produit $A(x)\x B(x)$
    
\end{enumerate}

\subsection*{Exercice 15}

Dresser le tableau de signes des expressions suivantes:
\begin{enumerate}
    \item $A(x)=(x+4)(2x-3)$
    \item $B(x)=(6-x)(x-3)$
\end{enumerate}



\subsection*{Exercice 16}

\begin{enumerate}
    \item Etudier le signe des expressions $A(x)=2-x$ et $B(x)=4x-3$
    \item En déduire le signe du produit $\dfrac{A(x)}{ B(x)}$
    
\end{enumerate}

\subsection*{Exercice 17}

Dresser le tableau de signes des expressions suivantes:
\begin{enumerate}
    \item $A(x)=\dfrac{6x-4}{2-3x}$
    \item $B(x)=\dfrac{3x+2}{5x-1}$
\end{enumerate}



\subsection*{Exercice 18}
\begin{enumerate}
\item Construire le tableau de signe du produit $(2x-4)(-3x-5)$.\medskip
\item Construire le tableau de signe du quotient $\frac{-3x+9}{7x-5}$.\medskip
\item Donner l'ensemble des solutions de l'inéquation $(2x-4)(-3x-5)<0$ grâce à la réponse à 1/.\medskip
\item Résoudre l'inéquation $\frac{-3x+9}{7x-5}\geqslant0$ grâce à 2/
\end{enumerate}



\subsection*{Exercice 19}

Dans un repère \Oij{}, on considère les points $A(-1;1)$, $B(2;2)$, $C(0;2)$ et $D(3;1)$.\\
On appelle $f$ la fonction affine dont la représentation graphique est la droite $(AB)$ et $g$ la fonction affine dont représentation graphique est la droite $(CD)$. On note $I$ le point d'intersection des droites $(AB)$ et$(CD)$.

\begin{enumerate}
    \item Placer les points $A$, $B$, $C$ et $D$ dans le repère \Oij{} puis tracer les droites $(AB)$ et $(CD)$
    \item Déterminer graphiquement une valeur approchée des coordonnées du point $I$
    \item Exprimer $f(x)$ en fonction de $x$ puis $g(x)$ en fonction de $x$
    \item Déterminer les coordonnées exactes de $I$
\end{enumerate}
\newpage
\subsection*{Exercice 20}

Deux chauffeurs de taxi pratiquent des tarifs  différents:

\begin{itemize}
    \item \textbf{Tarif A}: $5$ \EUR{} de prise en charge et $0,4$ \EUR{} par kilomètre parcouru.
    \item  \textbf{Tarif B}: Pas de frais de prise en charge et $0,6$ \EUR{} par kilomètre parcouru.
\end{itemize}

On désigne respectivement $A(x)$ et $ B(x)$ le prix payé en euros auprès des taxis $A$ et $B$ pour un nombre $x$ de kilomètres parcourus. \\
Dans le repère ci-dessous, on a représenté graphiquement les fonctions $x\mapsto A(x)$ et $x\mapsto B(x)$ pour $0\leq x\leq 40$



\begin{center}
\begin{tikzpicture}[scale=0.75]
\draw [gray,xstep=1,ystep=1] (-1,-1) grid (22,14);
\draw [->] (-1,0) -- (22,0) node [below] {km};
\draw [->] (0,-1) -- (0,14) node [left] {euros};

\draw (-0.2,0) node[left,below]{$0$};
\draw (1,0) node[below]{$2$};
\draw (5,0) node[below]{$10$};
\draw (10,0) node[below]{$20$};
\draw (15,0) node[below]{$30$};
\draw (20,0) node[below]{$40$};
\draw (0,1) node[left]{$2$};
\draw (0,5) node[left]{$10$};
\draw (0,10) node[left]{$20$};
\draw[domain=0:20] plot(\x,{0.6*(\x)} );
\draw[domain=0:20] plot(\x,{0.4*(\x)+2.5} );
\draw (4,4) node [above] {$\mathscr{C}_{1}$};
\draw (5,2) node [above] {$\mathscr{C}_{2}$};
\end{tikzpicture}
\end{center}

\begin{enumerate}
    \item Associer chacune des courbes $\mathscr{C}_{1}$ et $\mathscr{C}_{2}$ aux tarifs pratiqués par les taxis.
    \item Quel est le taxi le plus économique lorsque l'on désire faire un parcours de 20km ? de 30 km?
    \item Exprimer $A(x)$ et $B(x)$ en fonction de $x$.
    \item Un troisième taxi fait payer $11,50$\EUR{} pour $17$ km parcourus et $20,50$\EUR{} pour $35$ km.\\ Le prix d'un parcours est une fonction affine du nombre de kilomètres parcourus. \\ Déterminer le prix d'un parcours de $30 $ km avec ce taxi.
\end{enumerate}



\end{document}
 